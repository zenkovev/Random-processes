\section{Пуассоновский процесс}

\begin{definition}
    Случайный процесс $(X_t,\ t \ge 0)$ называется процессом с независимыми приращениями, если $\forall n \ \forall 0 \le t_1 < \dots < t_n$ $X_{t_n} - X_{t_{n-1}}, \dots, X_{t_2} - X_{t_1}, X_{t_1}$ -- независимые в совокупности случайные величины.
\end{definition}

\begin{definition}
    Случайный процесс $(N_t,\ t \ge 0)$ называется пуассоновским процессом интенсивности $\lambda > 0$, если выполнены три условия:
    \begin{enumerate}
        \item $N_0 = 0$ п.н.
        \item $N_t$ имеет независимые приращения
        \item $N_t - N_s \sim Pois(\lambda(t-s)) \ \forall t > s \ge 0$
    \end{enumerate}
\end{definition}

\begin{theorem} (без доказательства)
    Пусть $\{\xi_n\}_{n \in \N}$, $\xi_n \sim Exp(\lambda)$ -- независимые одинаково распределённые случайные величины. Тогда процесс восстановления $(N_t,\ t \ge 0)$, построенный по этим случайным величинам, является пуассоновским процессом интенсивности $\lambda$.
\end{theorem}

\begin{problem}
    Пусть $(N_t,\ t \ge 0)$ -- пуассоновский процесс интенсивности $\lambda$. Найти
    \[
        \frac{N_t}{t} \xrightarrow[t \to +\infty]{\text{п.н.}} \ ?
    \]
\end{problem}

\begin{note}
    Здесь сходимость почти наверное понимается в следующем смысле, это стоит оговорить, так как тут более чем счётное семейство случайных величин:
    \[
        \frac{N_t}{t} \xrightarrow[t \to +\infty]{\text{п.н.}} \xi \ \Lra \ P\ps{\omega \colon \frac{N_t}{t} \xrightarrow[t \to +\infty]{} \xi} = 1
    \]
\end{note}

\begin{solution}
    Начнём со счётной последовательности случайных величин $N_0, N_1, N_2, \dots$. По определению пуассоновского процесса случайные величины $N_1 - N_0, \dots, N_n - N_{n-1}$ независимы, следовательно, вся последовательность $N_1 - N_0, N_2 - N_1, \dots$ случайных величин независима. Также, $N_0 = 0$ п.н., а $N_n - N_{n-1} \sim Pois(\lambda(n - (n-1))) = Pois(\lambda)$. Тогда, применяя УЗБЧ, получим:
    \[
        \frac{N_n}{n} = \frac{N_0 + (N_1-N_0) + \dots (N_n-N_{n-1})}{n} \xrightarrow{\text{п.н.}} \E Pois(\lambda) = \lambda
    \]

    Теперь заметим, что так как для $t > s$ выполнено $N_t - N_s \sim Pois(\lambda(t-s))$, а распределение Пуассона принимает только целые неотрицательные значения, то при всех $\omega$ функция $N_t(\omega)$ не убывает по $t$, тогда можем сделать оценки при всех $\omega$ и при всех $t$:
    \[
        \frac{N_{[t]}}{[t]} \frac{[t]}{[t]+1} = \frac{N_{[t]}}{[t]+1} \le \frac{N_t}{t} \le \frac{N_{[t]+1}}{[t]} = \frac{N_{[t]+1}}{[t]+1} \frac{[t]+1}{[t]}
    \]

    Зафиксируем $\omega$, такое, что $\frac{N_n(\omega)}{n} \xrightarrow[n \to \infty]{} \lambda$, как уже выяснили, все подходящие $\omega$ образуют множество вероятности 1. При таком фиксированном $\omega$ в последнем неравенстве, так как $[t]$ и $[t]+1$ целые, можем применить счётный предел, помимо этого, воспользуемся простыми соображениями из курса математического анализа:
    \begin{align*}
        & \lim_{t \to +\infty} \frac{N_{[t]}(\omega)}{[t]} = \lim_{t \to +\infty} \frac{N_{[t]+1}(\omega)}{[t]+1} = \lambda
        \\
        & \lim_{t \to +\infty} \frac{[t]}{[t]+1} = \lim_{t \to +\infty} \frac{[t]+1}{[t]} = 1
        \\
        & \Downarrow
        \\
        & \frac{N_t(\omega)}{t} \xrightarrow[t \to +\infty]{} \lambda
    \end{align*}
    
    Так как мы брали $\omega$ из множества единичной вероятности, то мы доказали, что:
    \[
        \frac{N_t}{t} \xrightarrow[t \to +\infty]{\text{п.н.}} \lambda
    \]

    Смысл полученного результата следующий: так как процесс восстановления для экспоненциальных случайных величин является пуассоновским процессом, а $N_t$ в таком случае есть количество починок прибора за время $t$, то за большой промежуток времени мы будем чинить прибор со скоростью $\approx \lambda$ раз/ед.врем.
\end{solution}

\begin{definition}
    Случайный процесс $(Y_t,\ t \in T)$ называется модификацией случайного процесса $(X_t,\ t \in T)$, если $\forall t \in T$ выполнено $P(X_t = Y_t) = 1$.
\end{definition}

\begin{example}
    Пусть $(\Omega, \F, P) = ([0, 1], \B[0, 1], \lambda)$, где $\lambda$ -- классическая мера Лебега на $[0, 1]$, $T = [0, 1]$.  Рассмотрим случайные процессы:
    \begin{align*}
        & X_t \equiv 0
        \\
        & Y_t = \System{
            & 0,\ t \neq \omega,
            \\
            & 1,\ t = \omega
        }
    \end{align*}

    Процессы $X_t$ и $Y_t$ являются модификациями друг друга, но при этом у них не совпадает ни одна траектория.
\end{example}

\begin{problem}
    Доказать, что у пуассоновского процесса $(N_t,\ t \ge 0)$ интенсивности $\lambda$ не существует непрерывной модификации, то есть модификации $(X_t,\ t \ge 0)$, у которой все траектории $X_t$ непрерывны по $t$.
\end{problem}

\begin{proof}
    Предположим противное: пусть $(X_t, t \ge 0)$ -- непрерывная модификация пуассоновского процесса $(N_t, t \ge 0)$.

    Рассмотрим счётное множество $\Q \cap [0, 1]$ как подмножество моментов времени. Возьмём момент времени $q \in \Q \cap [0, 1]$, тогда $N_q \sim N_q - N_0 \sim Pois(\lambda q)$ в силу свойств 1 и 3 из определения пуассоновского процесса, а так как $P(X_q = N_q) = 1$, то $X_q \sim Pois(\lambda q)$. Здесь носитель $X_q$ уже не обязательно целый, но так как распределение Пуассона принимает целые значения, то множество $A_q = \{\omega \colon X_q \in \Z\}$ имеет единичную вероятность $P(A_q) = 1$.

    В таком случае рассмотрим множество $A = \bigcap\limits_{q \in \Q \cap [0, 1]} A_q$, так как это счётное пересечение множеств единичной вероятности, то $P(A) = 1$. При этом выполнено:
    \[
        \forall \omega \in A \ \forall q \in \Q \cap [0, 1] \ \ X_q = X_q(\omega) \in \Z
    \]

    Зафиксируем $\omega \in A$. Знаем, что $X_t(\omega)$ непрерывна по $t$, при этом в точках из $\Q \cap [0, 1]$ $X_t(\omega)$ принимает только целые значения. Так как множество $\Z$ замкнуто, а по непрерывности $X_t(\omega)$ однозначно восстанавливается на $[0, 1]$ по своим значениям в рациональных точках, то $X_t(\omega)$ на всём $[0, 1]$ принимает только целые значения, значит, $X_t(\omega) \equiv const$ на $[0, 1]$.

    Так как $N_0 = 0$ п.н., $X_0 = N_0$ п.н., то $X_0 = 0$ п.н., тогда множество $\{\omega \colon X_0 = 0\} \cap A$ имеет вероятность 1, при этом, так как доказали постоянность траекторий на отрезке $[0, 1]$ на множестве $A$:
    \[
        \forall \omega \in \{\omega \colon X_0 = 0\} \cap A \ \forall t \in [0, 1] \ \ X_t = X_t(\omega) = 0 
    \]

    Отсюда следует, что $X_{1/2} = 0$ п.н., в то же время $X_{1/2} \sim Pois\ps{\frac{1}{2} \lambda}$, получили противоречие, что нам и было нужно.    
\end{proof}

\begin{problem}
    Пусть $X = (X_t,\ t \in [0, T])$ -- случайный процесс, для которого $P(X_s \le X_t) = 1$ при $0 \le s \le t \le T$. Доказать, что у $X$ существует модификация, имеющая почти наверное неубывающие траектории.
\end{problem}

\begin{note}
    В частности, условие задачи выполнено для пуассоновского процесса.
\end{note}

\begin{proof}~
    \begin{itemize}
        \item[Шаг 1] Докажем, что $\forall t \in [0, T] \ \ \exists P\text{-}\lim\limits_{s \to t-0} X_s \text{ и } P\text{-}\lim\limits_{s \to t+0} X_s$, то есть пределы слева и справа в точке $t$ случайных величин $X_s$ по вероятности, докажем для предела справа, для предела слева доказывается аналогично. То есть нужно доказать, что для фиксированного $t \in [0, T]$ существует случайная величина $X_{t+}(\omega)$, являющаяся пределом справа по вероятности, иными словами, выполнено:
        \[
            \forall \eps > 0 \ \ P(|X_s - X_{t+}| \ge \eps) \xrightarrow[s \to t+0]{} 0
        \]

        Так как с последовательностью случайных величин работать проще, чем с их континуальным семейством, сначала разберёмся со счётным случаем. Пусть $t_n \downarrow t$, то есть возьмём произвольную последовательность $t_n$, такую, что $t_n$ убывает и сходится к $t$. Так как, в силу условия, почти наверное $X_{t_1} \ge X_{t_2}$, почти наверное $X_{t_2} \ge X_{t_3}$ и так далее, то счётное пересечение этих событий тоже имеет вероятность 1, то есть почти наверное $X_{t_1} \ge X_{t_2} \ge X_{t_3} \ge \dots$. Аналогично, почти наверное $X_{t_1} \ge X_t$, почти наверное $X_{t_2} \ge X_t$ и так далее, переходим к счётному пересечению, получаем, что почти наверное $X_{t_1} \ge X_{t_2} \ge X_{t_3} \ge \dots \ge X_t$. В таком случае, почти наверное последовательность $X_{t_n}$ убывает и ограничена снизу через $X_t$, следовательно, почти наверное существует и конечен предел $X_{t+} = \lim\limits_{n \to \infty} X_{t_n}$. Более того, $X_{t+}$, как предел счётной последовательности случайных величин, является случайной величиной. И так как сходимость почти наверное влечёт сходимость по вероятности, получим:
        \[
            \forall \eps > 0 \ \ P\ps{|X_{t_n} - X_{t+}| \ge \eps} \xrightarrow[n \to \infty]{} 0
        \]
        
        Теперь хотим от предела по последовательности $t_n$ перейти к пределу по $s$. Для этого нам будет достаточно такой оценки:
        \[
            \forall t < s_1 \le s_2 \ \ P\ps{|X_{s_2} - X_{t+}| \ge \eps} \ge P\ps{|X_{s_1} - X_{t+}| \ge \eps}
        \]

        Действительно, если такая оценка верна, то последовательность $P\ps{|X_s - X_{t+}| \ge \eps}$ при $s \to t+0$ убывает, конечно, ограничена снизу, поэтому сходится, и предел будет таким же, что и предел по последовательности $t_n$:
        \[
            \lim_{s \to t+0} P\ps{|X_s - X_{t+}| \ge \eps} = \lim_{n \to \infty} P\ps{|X_{t_n} - X_{t+}| \ge \eps} = 0
        \]

        Поняли, что такой оценки достаточно, теперь докажем эту самую оценку. Вспомним, что по условию при $s_1 \le s_2$ выполнено $P(X_{s_1} \le X_{s_2}) = 1$. Заметим ещё два факта, во-первых, $P(X_{t+} \ge X_t) = 1$, во-вторых, при $s > t$ выполнено $P(X_s \ge X_{t+}) = 1$. Первый факт следует из того, что почти наверное $X_{t_1} \ge X_{t_2} \ge \dots \ge X_t$ и $X_{t+} = \lim\limits_{n \to \infty} X_{t_n}$. Второй факт следует из того, что имеем $t < s$, выберем $t_n \in (t, s)$, получим, что почти наверное $X_{t_n} \ge X_{t+}$ в силу определения $X_{t+}$ и почти наверное $X_s \ge X_{t_n}$ в силу условия.
        
        Теперь вернёмся к оценке, которую хотим доказать, пусть $t < s_1 \le s_2$. Продолжим наблюдения, которые дадут итоговый результат:
        \begin{align*}
            P(X_{s_1} \ge X_{t+}) = 1 \ \Ra \ & P\ps{|X_{s_1} - X_{t+}| \ge \eps} = P\ps{|X_{s_1} - X_{t+}| \ge \eps,\ X_{s_1} \ge X_{t+}} =
            \\
            & = P\ps{X_{s_1} - X_{t+} \ge \eps}
            \\
            P(X_{s_2} \ge X_{t+}) = 1 \ \Ra \ & P\ps{|X_{s_2} - X_{t+}| \ge \eps} = P\ps{X_{s_2} - X_{t+} \ge \eps}
            \\
            P(X_{s_1} \le X_{s_2}) = 1 \ \Ra \ & P\ps{X_{s_1} - X_{t+} \ge \eps} = P\ps{X_{s_1} - X_{t+} \ge \eps,\ X_{s_1} \le X_{s_2}} \le
            \\
            & \le P\ps{X_{s_2} - X_{t+} \ge \eps}
        \end{align*}

        Итоговый результат, действительно, получили:
        \[
            P\ps{|X_{s_1} - X_{t+}| \ge \eps} = P\ps{X_{s_1} - X_{t+} \ge \eps} \le P\ps{X_{s_2} - X_{t+} \ge \eps} = P\ps{|X_{s_2} - X_{t+}| \ge \eps}
        \]

        Хорошо, получили, что при всех $t \in [0, T]$ существуют пределы:
        \[
            P\text{-}\lim\limits_{s \to t-0} X_s = X_{t-} \text{ и } P\text{-}\lim\limits_{s \to t+0} X_s = X_{t+}
        \]

        \item[Шаг 2] Докажем, что для всех $t \in [0, T]$, кроме, возможно, счётного числа, выполнено почти наверное равенство $P(X_{t-} = X_t = X_{t+}) = 1$. Сначала вспомним несколько фактов, полученных при доказательстве предыдущего шага, пусть $s < t$, возьмём $r \in (s, t)$:
        \begin{align*}
            & \text{п.н. } X_{s-} \le X_s \le X_{s+}
            \\
            & \text{п.н. } X_{t-} \le X_t \le X_{t+}
            \\
            & \text{п.н. } X_{s+} \le X_r \le X_{t-}
            \\
            & \Downarrow
            \\
            & \text{п.н. } X_{s-} \le X_s \le X_{s+} \le X_{t-} \le X_t \le X_{t+}
        \end{align*}

        Пока предположим, что при всех $t \in [0, T]$ случайные величины $X_t$, $X_{t+}$, $X_{t-}$ имеют математические ожидания. Для произвольных $s < t,\ s, t \in [0, T]$ в силу полученного только что факта:
        \begin{align*}
            & X_{s-} \le X_s \le X_{s+} \le X_{t-} \le X_t \le X_{t+} \text{ почти наверное}
            \\
            & \Downarrow
            \\
            & \E X_{s-} \le \E X_s \le \E X_{s+} \le \E X _{t-} \le \E X_t \le \E X_{t+}
        \end{align*}

        Теперь зачем всё это нужно: рассмотрим плохую точку $t$, в которой утверждение шага не выполнено, то есть $P(X_{t-} = X_t = X_{t+}) < 1$, мы знаем $P(X_{t-} \le X_t \le X_{t+}) = 1$, тогда возможно два варианта: либо $P(X_{t-} < X_t) > 0$, либо $P(X_t < X_{t+}) > 0$, отсюда следует:
        \begin{align*}
            & P(X_{t-} \le X_t) = 1,\ P(X_{t-} < X_t) > 0 \ \Ra \ \E X _{t-} < \E X_t
            \\
            & \text{либо}
            \\
            & P(X_t \le X_{t+}) = 1,\ P(X_t < X_{t+}) > 0 \ \Ra \ \E X_t < \E X_{t+}
            \\
            & \Downarrow
            \\
            & \E X _{t-} < \E X_{t+}
        \end{align*}

        Тогда проблемной точке $t$ можно сопоставить рациональное число $q \in (\E X _{t-},\ \E X_{t+})$, и в силу того, что при $s < t$ выполнено $\E X_{s-} \le \E X_{s+} \le \E X _{t-} \le \E X_{t+}$, такое число будет уникальным для каждой проблемной точки. Это доказывает, что мы можем пересчитать все плохие точки, то есть их действительно счётное число.

        Теперь поймём, что мир устроен немного сложнее, и конечные математические ожидания существуют далеко не всегда. Тем не менее, общий случай можно свести к только что рассмотренному частному, для этого нам понадобятся срезки случайных величин:
        \[
            X_{t,[N]}(\omega) = \System{
                & -N,\ X_t(\omega) \le -N,
                \\
                & X_t(\omega),\ -N < X_t(\omega) < N,
                \\
                & N,\ X_t(\omega) \ge N
            }
        \]

        Для срезок тоже можем рассмотреть множество плохих точек:
        \[
            T_N = \set{t \in [0, T] \colon P(X_{t-,[N]} = X_{t,[N]} = X_{t+,[N]}) < 1}
        \]

        Срезки наследуют полученные выше неравенства вероятности 1, при этом в силу ограниченности они имеют конечные математические ожидания, поэтому для них работает последнее рассуждение, и множество проблемных точек $T_N$ счётно. Счётное объединение счётных множеств счётно, поэтому $T_\infty = \bigcup\limits_{N=1}^\infty T_N$ тоже является счётным множеством. Оказывается, что $T_\infty$ содержит все проблемные точки исходных случайных величин.

        Действительно, возьмём $t \notin T_\infty$, то есть для $t$ выполнено:
        \[
            \forall N \in \N \ \ P(X_{t-,[N]} = X_{t,[N]} = X_{t+,[N]}) = 1
        \]

        Можем записать теоретико-множественные равенства:
        \begin{multline*}
            \{\omega: X_{t-} = X_t = X_{t+}\} = \{\omega \colon \forall N \in \N \ \ X_{t-,[N]} = X_{t,[N]} = X_{t+,[N]}\} =
            \\
            = \bigcap_{N=1}^\infty \{\omega \colon X_{t-,[N]} = X_{t,[N]} = X_{t+,[N]}\}
        \end{multline*}

        В силу вложенности множеств из последнего пересечения можем применить непрерывность вероятностной меры:
        \[
            P(X_{t-} = X_t = X_{t+}) = \lim_{N \to \infty} P(X_{t-,[N]} = X_{t,[N]} = X_{t+,[N]}) = \lim_{N \to \infty} 1 = 1
        \]
            
        То есть получили, что если $t \notin T_\infty$, то она не является проблемной, следовательно, в $T_\infty$ содержатся все проблемные точки, следовательно, множество проблемных точек счётно, а это в точности утверждение текущего шага.

        \item[Шаг 3] Рассмотрим множество $T_0 = T_\infty \cup (\Q \cap [0, T])$. Это множество, во-первых, счётно, как объединение двух счётных множеств, во-вторых, всюду плотно в $[0, T]$, так как содержит рациональные точки отрезка, в-третьих, $\forall t \notin T_0 \ \ P(X_{t-} = X_t = X_{t+}) = 1$.

        Теперь наконец-то определим модификацию, то, что получится модификация, докажем ниже, случайного процесса $X_t$:
        \[
            Y_t(\omega) = \System{
                & X_t(\omega),\ t \in T_0,
                \\
                & \inf_{s > t,\ s \in T_0} X_s(\omega),\ t \notin T_0
            }
        \]

        Посмотрим, как устроен случайный процесс $(Y_t,\ t \in [0, T])$. Для этого посмотрим на случайный процесс $(X_t,\ t \in [0, T])$ в счётном множестве точек $T_0$:
        \begin{align*}
            & P(X_s \le X_t) = 1 \ \ \forall s \le t,\ s, t \in T_0
            \\
            & \Downarrow
            \\
            & P(X_s \le X_t \ \ \forall s \le t,\ s, t \in T_0) = P\ps{\bigcap_{s \le t,\ s, t \in T_0} X_s \le X_t} = 1,
            \\
            & \text{так как пар } s \le t,\ s, t \in T_0 \text{ счётное число}
        \end{align*}

        При этом по построению на множестве $\{\omega \colon X_s \le X_t \ \ \forall s \le t,\ s, t \in T_0\}$ траектории случайного процесса $Y_t$ являются неубывающими, то есть почти все траектории случайного процесса $Y_t$ являются неубывающими.

        Для завершения доказательства нам остаётся понять, почему случайный процесс $Y_t$ является модификацией случайного процесса $X_t$.

        Зафиксируем $t \in T$, нужно доказать, что $P(X_t = Y_t) = 1$. Если $t \in T_0$, то доказывать нечего, по построению $X_t \equiv Y_t$, таким образом, пусть $t \notin T_0$.  Заметим, что для $\omega$ из множества $\{\omega \colon X_{s_1} \le X_{s_2} \ \ \forall s_1 \le s_2,\ s_1, s_2 \in T_0\}$, которое имеет единичную вероятность, в силу неубывания инфимум вырождается в предел справа:
        \[
            Y_t(\omega) = \inf_{s > t,\ s \in T_0} X_s(\omega) = \lim_{s \to t+0,\ s \in T_0} X_s(\omega)
        \]

        В частности, если выберем $\{t_n\}_{n \in \N} \subset T_0$, $t_n \downarrow t$, то получим:
        \begin{align*}
            & Y_t(\omega) = \lim_{n \to \infty} X_{t_n}(\omega) \ \ \forall \omega \in \{\omega \colon X_{s_1} \le X_{s_2} \ \ \forall s_1 \le s_2,\ s_1, s_2 \in T_0\}
            \\
            & \Downarrow
            \\
            & Y_t = \lim_{n \to \infty} X_{t_n} \text{ почти наверное}
            \\
            & \Downarrow
            \\
            & X_{t_n} \xrightarrow{P} Y_t
        \end{align*}

        Но вспомним, что уже находили предел по вероятности $X_{t+} = P\text{-}\lim\limits_{s \to t+0} X_s$, который, в частности, является пределом по вероятности $X_{t_n} \xrightarrow{P} X_{t+}$. В силу теоремы Рисса, из последовательности, сходящейся по вероятности, можно выделить подпоследовательность, сходящуюся почти наверное, что мгновенно означает, что предел по вероятности единственен с точностью до множества нулевой вероятности.
        
        Отсюда и из того, что $t \notin T_0$, следует, что $Y_t = X_{t+} = X_t \text{ почти наверное}$. Последнее означает, что $P(X_t = Y_t) = 1$, а это ровно то, что нам оставалось доказать.
    \end{itemize}
\end{proof}