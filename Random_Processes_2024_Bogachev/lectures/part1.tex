\section{Задачи про компактные классы}

\begin{definition}
    Класс $\cK$ подмножеств $X$ называется компактным, если для всякой последовательности множеств из $\cK$ с пустым пересечением есть конечный поднабор множеств с пустым пересечением, то есть:
    \[
        \bigcap_{n=1}^\infty K_n = \emptyset,\ K_n \in \cK \ \Ra \ \exists N \ \bigcap_{n=1}^N K_n = \emptyset
    \]
\end{definition}

\begin{problem}~
    \begin{enumerate}
        \item Доказать, что любой набор компактов в $\R^n$ -- компактный класс.

        \item Доказать, что совокупность всех цилиндров в $\R^T$ вида $C_{t_1, \dots, t_n, K}$, где $K$ -- компакт в $\R^n$, есть компактный класс.

        \item Пусть $P$ -- аддитивная неотрицательная функция на алгебре $\cA$, причём имеется такой компактный класс $\cK \subset \cA$, что для всякого $A \in \cA$ выполнено:
        \[
            P(A) = \sup_{K \subset A,\ K \in \cK} P(K)
        \]
        Доказать, что функция $P$ счётно-аддитивна.
    \end{enumerate}
\end{problem}

\begin{solution}~
    \begin{enumerate}
        \item Докажем двумя способами, доказательства будут ссылаться на различные эквивалентные определения компактности в $\R^n$ и будут иметь различные обобщения на случай произвольных топологических пространств.

        \begin{itemize}
            \item Пусть $X = \R^n$ -- объемлющее пространство. Рассмотрим произвольную последовательность компактов $K_n$, имеющих пустое пересечение $\bigcap_{n=1}^\infty K_n = \emptyset$. С помощью теоретико-множественных фокусов получим:
            \[
                \bigcap_{n=1}^\infty K_n = \emptyset \ \Ra \ \bigcup_{n=1}^\infty \ol{K_n} = X \ \Ra \ K_1 \subset \bigcup_{n=1}^\infty \ol{K_n}
            \]
            
            Вспомним, что компакт $K$ является замкнутым множеством, следовательно, его дополнение $\ol{K}$ открыто. Воспользуемся определением компакта, что из любого открытого покрытия можно выделить конечное подпокрытие:
            \[
                K_1 \subset \bigcup_{n=1}^\infty \ol{K_n} \ \Ra \ \exists N \ K_1 \subset \bigcup_{n=1}^N \ol{K_n}
            \]

            И отсюда выведем требуемое:
            \[
                K_1 \subset \bigcup_{n=1}^N \ol{K_n} \ \Ra \ \bigcap_{n=1}^N K_n \subset \ol{K_1} \ \Ra \ K_1 \cap \bigcap_{n=1}^N K_n = \bigcap_{n=1}^N K_n = \emptyset
            \]

            Это рассуждение проходит для всех топологических пространств $X$, в которых из компактности следует замкнутость.

            \item Пусть $X = \R^n$ -- объемлющее пространство. Рассмотрим произвольную последовательность компактов $K_n$, предположим, что для всех $N$ $\bigcap_{n=1}^N K_n \neq \emptyset$, докажем, что в таком случае $\bigcap_{n=1}^\infty K_n \neq \emptyset$.

            Так как $\bigcap_{n=1}^N K_n \neq \emptyset$, то существует $x_N \in \bigcap_{n=1}^N K_n$. Получим последовательность $\{x_N\}_{N=1}^\infty$, по построению она полностью лежит в компакте $K_1$. В силу компактности $K_1$ из неё можно выделить сходящуюся в $K_1$ подпоследовательность, то есть $\{x_{N_k}\}_{k=1}^\infty \subset \{x_N\}_{N=1}^\infty$, $x_{N_k} \to x \in K_1$.
            
            Теперь зафиксируем произвольное $m \in \N$. Так как последовательность $\{x_{N_k}\}_{k=1}^\infty$, по построению, начиная с некоторого номера, полностью лежит в компакте $K_m$, при этом она сходится $x_{N_k} \to x$, то предел тоже лежит в $K_m$, то есть $x \in K_m$. Отсюда следует, что $x \in \bigcap_{n=1}^\infty K_n$, то есть это пересечение непусто, что и хотели доказать.

            Это рассуждение проходит для всех топологических пространств $X$, для класса секвенциально компактных множеств.
        \end{itemize}

        \item Сначала докажем, что конечное пересечение цилиндров с компактными основаниями является цилиндром с компактным основанием, достаточно доказать для двух. Действительно, пусть есть два цилиндра $C_{s_1, \dots, s_m, K_1}$, $C_{t_1, \dots, t_n, K_2}$, где $K_1$, $K_2$ -- компакты в $\R^m$ и $\R^n$ соответственно. Их и их пересечение можно записать в следующем виде, не важно, что среди моментов времени $s_1, \dots, s_m, t_1, \dots, t_n$ могут быть дублирующиеся:
        \begin{align*}
            & C_{s_1, \dots, s_m, K_1} = \{x \in \R^T \colon (x(s_1), \dots, x(s_m), x(t_1), \dots, x(t_n)) \in K_1 \times \R^n\}
            \\
            & C_{t_1, \dots, t_n, K_2} = \{x \in \R^T \colon (x(s_1), \dots, x(s_m), x(t_1), \dots, x(t_n)) \in \R^m \times K_2\}
            \\
            & C_{s_1, \dots, s_m, K_1} \cap C_{t_1, \dots, t_n, K_2} = \{x \in \R^T \colon (x(s_1), \dots, x(s_m), x(t_1), \dots, x(t_n)) \in K_1 \times K_2\}
        \end{align*}

        Изначально, моменты времени $s_1, \dots, s_m$ были различны, моменты времени $t_1, \dots, t_n$ тоже были различны. Теперь возможное дублирование моментов времени внутри набора $s_1, \dots, s_m, t_1, \dots, t_n$ придётся победить, для этого будет немного формализма. Выберем дублирования из $s$-ок и $t$-шек, обозначим через $r$-ки, при необходимости поменяем нумерацию $s$-ок и $t$-шек, чтобы $r$-ки шли в конце.
        
        Иными словами, без ограничения общности, выполнено:
        \begin{align*}
            & (s_1, \dots, s_m, t_1, \dots, t_n) = (s_1, \dots, s_i, r_1, \dots, r_k, t_1, \dots, t_j, r_1, \dots, r_k)
            \\
            & s_1, \dots, s_i, t_1, \dots, t_j, r_1, \dots, r_k \text{ -- различные элементы } T
        \end{align*}
        
        Теперь хотим записать $K_1 \times K_2$ из пересечения цилиндров в другом виде для моментов времени без дублирования $s_1, \dots, s_i, t_1, \dots, t_j, r_1, \dots, r_k$. Для этого рассмотрим следующее множество:
        \begin{multline*}
            K = \{(S_1, \dots, S_i, T_1, \dots, T_j, R_1, \dots, R_k) \in \R^{i+j+k} \colon
            \\
            (S_1, \dots, S_i, R_1, \dots, R_k, T_1, \dots, T_j, R_1, \dots, R_k) \in K_1 \times K_2\}
        \end{multline*}
            
        Множество $K$ ограничено, так как в силу ограниченности $K_1$ и $K_2$ все координаты произвольной точки из $K$ ограничены. Множество $K$ замкнуто, так как если есть предел у последовательности векторов вида $(S_1, \dots, S_i, T_1, \dots, T_j, R_1, \dots, R_k)$, зависимость от порядкового номера в последовательности здесь не указана, то он даёт пределы для $(S_1, \dots, S_i, R_1, \dots, R_k)$ и $(T_1, \dots, T_j, R_1, \dots, R_k)$, а эти пределы не покидают компакты $K_1$ и $K_2$. Множество $K$ замкнуто и ограничено, то есть компактно, так как мы живём в $\R^{i+j+k}$.
        
        Отсюда мгновенно следует, что пересечение двух данных цилиндров с компактными основаниями тоже есть цилиндр с компактным основанием:
        \begin{multline*}
            C_{s_1, \dots, s_m, K_1} \cap C_{t_1, \dots, t_n, K_2} = \{x \in \R^T \colon (x(s_1), \dots, x(s_m), x(t_1), \dots, x(t_n)) \in K_1 \times K_2\} =
            \\
            = \{x \in \R^T \colon (x(s_1), \dots, x(s_i), x(t_1), \dots, x(t_j), x(r_1), \dots, x(r_k)) \in K\}
        \end{multline*}

        Таким образом, получили, что конечное пересечение цилиндров с компактными основаниями тоже есть цилиндр с компактным основанием. Хорошо, с этим справились.

        Теперь обратим внимание на ещё один полезный факт, что из вложения цилиндров с компактными основаниями следует обратная вложенность соответствующих моментов времени, имеется в виду, что:
        \begin{align*}
            & C_{s_1, \dots, s_m, K_1} \subset C_{t_1, \dots, t_n, K_2},\ K_1, K_2 \text{ -- компакты в } \R^m \text{ и } \R^n
            \\
            & \Downarrow
            \\
            & \{t_1, \dots, t_n\} \subset \{s_1, \dots, s_m\}
        \end{align*}

        Это логично, так как означает, что меньший цилиндр учитывает все ограничения большего цилиндра, возможно, добавляя к ним какие-то новые. Формально доказывается от противного: пусть $t_i \in \{t_1, \dots, t_n\} \setminus \{s_1, \dots, s_m\}$, возьмём $x \in C_{s_1, \dots, s_m, K_1}$ и переопределим $x(t_i)$, получив функцию $y \in \R^T$, так, чтобы $(y(t_1), \dots, y(t_n))$ гарантированно не попадала в компакт $K_2$, так можно сделать, ибо все координаты точек из компакта ограничены. Так как $x$ и $y$ отличаются только в точке $t_i \notin \{s_1, \dots, s_m\}$ и $x \in C_{s_1, \dots, s_m, K_1}$, то и $y \in C_{s_1, \dots, s_m, K_1}$. При этом по выбору $y(t_i)$ выполнено $y \notin C_{t_1, \dots, t_n, K_2}$. Противоречие с $C_{s_1, \dots, s_m, K_1} \subset C_{t_1, \dots, t_n, K_2}$.

        Справились с двумя вспомогательными фактами, теперь уже можно приступить к основному доказательству.
        
        Нужно доказать, что совокупность всех цилиндров с компактными основаниями есть компактный класс. Рассмотрим последовательность цилиндров с компактными основаниями $C_1, C_2, \dots$, таких, что:
        \[
            \forall n \in \N \ \bigcap_{k=1}^n C_k \neq \emptyset
        \]
        Нужно доказать, что тогда пересечение всех этих цилиндров тоже не пусто. Для этого также рассмотрим последовательность $D_1, D_2, \dots$, где $D_n = \bigcap_{k=1}^n C_k$, она является последовательностью непустых вложенных цилиндров с компактными основаниями, непустота по предположению, вложенность по построению, компактность основания в силу уже доказанного. Элементы этой последовательности можно записать в виде:
        \begin{align*}
            & D_1 = \{x \in \R^T \colon (x(t_1), \dots, x(t_{d_1})) \in K_1\}
            \\
            & D_2 = \{x \in \R^T \colon (x(t_1), \dots, x(t_{d_1}), \dots, x(t_{d_2})) \in K_2\}
            \\
            & D_3 = \{x \in \R^T \colon (x(t_1), \dots, x(t_{d_1}), \dots, x(t_{d_2}), \dots, x(t_{d_3})) \in K_3\}
            \\
            & \vdots
        \end{align*}
        Здесь $K_1, K_2, \dots$ -- компакты, $t_1, t_2, \dots$ -- конечная или счётная последовательность моментов времени. То, что моменты времени можно записать именно в таком виде, следует из вложенности $D_1 \supset D_2 \supset \dots$, влекущей обратную вложенность соответствующих моментов времени, говоря иначе, каждый следующий цилиндр $D_{n+1}$ наследует все моменты времени предыдущего цилиндра $D_n$, возможно, добавляя к ним новые.

        Теперь воспользуемся непустотой цилиндров $D_n$, а именно выберем последовательность точек, они будут иметь разную размерность, из компактов $K_n$:
        \begin{align*}
            & y_1^{(0)} = (y_{1, 1}^{(0)}, \dots, y_{1, d_1}^{(0)}) \in K_1
            \\
            & y_2^{(0)} = (y_{2, 1}^{(0)}, \dots, y_{2, d_1}^{(0)}, y_{2, d_1+1}^{(0)}, \dots, y_{2, d_2}^{(0)}) \in K_2
            \\
            & y_3^{(0)} = (y_{3, 1}^{(0)}, \dots, y_{3, d_1}^{(0)}, y_{3, d_1+1}^{(0)}, \dots, y_{3, d_2}^{(0)}, y_{3, d_2+1}^{(0)}, \dots, y_{3, d_3}^{(0)}) \in K_3
            \\
            & \vdots
        \end{align*}

        Посмотрим на первые префиксы этих точек $(y_{1, 1}^{(0)}, \dots, y_{1, d_1}^{(0)}), (y_{2, 1}^{(0)}, \dots, y_{2, d_1}^{(0)}), \dots$. Так как можно определить функции $x_1, x_2, \dots \in \R^T$, такие, что
        \[
            (x_k(t_1), \dots, x_k(t_{d_k})) = (y_{k, 1}^{(0)}, \dots, y_{k, {d_k}}^{(0)}) \ \Ra \ x_k \in D_k
        \]
        и есть вложенность $D_1 \supset D_2 \supset \dots$, то выполнено $x_1, x_2, \dots \in D_1$, откуда все первые префиксы $(y_{1, 1}^{(0)}, \dots, y_{1, d_1}^{(0)}), (y_{2, 1}^{(0)}, \dots, y_{2, d_1}^{(0)}), \dots$ лежат в $K_1$. Так как $K_1$ является компактом, то из полученной последовательности первых префиксов можно выделить сходящуюся в $K_1$ подпоследовательность. Иными словами, из последовательности $y_1^{(0)}$, $y_2^{(0)}, \dots$ можно выделить подпоследовательность $y_1^{(1)}$, $y_2^{(1)}, \dots$, такую, что у неё существует предел по первому префиксу $\lim\limits_{k \to \infty} (y_{k, 1}^{(1)}, \dots, y_{k, {d_1}}^{(1)})$, который лежит в $K_1$.

        Теперь смотрим на новую последовательность точек разной размерности $y_1^{(1)}$, $y_2^{(1)}, \dots$, начиная с некоторого номера, у элементов этой последовательности определены вторые префиксы $(y_{k, 1}^{(1)}, \dots, y_{k, d_2}^{(1)})$. По тем же причинам, что и раньше, все вторые префиксы лежат в компакте $K_2$. Опять выбираем подпоследовательность $y_1^{(2)}$, $y_2^{(2)}, \dots$ текущей последовательности $y_1^{(1)}$, $y_2^{(1)}, \dots$, такую, что у неё существует предел по второму префиксу $\lim\limits_{k \to \infty} (y_{k, 1}^{(2)}, \dots, y_{k, {d_2}}^{(2)})$, который лежит в $K_2$.

        Продолжая далее эту процедуру, выбираем последовательность $y_1^{(n)}$, $y_2^{(n)}, \dots$, такую, что у неё существует предел по $n$-му префиксу $\lim\limits_{k \to \infty} (y_{k, 1}^{(n)}, \dots, y_{k, {d_n}}^{(n)})$, лежащий в $K_n$.

        Теперь применим классический диагональный метод и рассмотрим последовательность точек разной размерности $y_1^{(1)}, y_2^{(2)}, y_3^{(3)}, \dots$. У неё существует предел по $n$-му префиксу для всех натуральных $n$. Более того, если $\lim\limits_{k \to \infty} (y_{k, 1}^{(n)}, \dots, y_{k, {d_n}}^{(n)}) = (z_1, \dots, z_{d_n})$, то для всех $m \le n$ предел $\lim\limits_{k \to \infty} (y_{k, 1}^{(m)}, \dots, y_{k, {d_m}}^{(m)}) = (z_1, \dots, z_{d_m})$, в силу того, что векторная сходимость влечёт покомпонентную.

        Отсюда получаем конечную или счётную последовательность точек $z_1, z_2, \dots$ из $\R$, такую, что при всех $n$ выполнено $(z_1, \dots, z_{d_n}) \in K_n$, так как этот префикс является пределом $\lim\limits_{k \to \infty} (y_{k, 1}^{(n)}, \dots, y_{k, {d_n}}^{(n)}) = \lim\limits_{k \to \infty} (y_{k, 1}^{(k)}, \dots, y_{k, {d_n}}^{(k)})$, который, как было замечено ранее, лежит в $K_n$.

        Остался последний шаг: определим $x \in \R^T$ как $x(t_1) = z_1,\ x(t_2) = z_2, \dots$, в оставшихся точках из $T$ доопределим произвольным образом. Тогда по построению и формуле для множеств $D_n$ выполнено $x \in D_1,\ x \in D_2, \dots$, иными словами:
        \[
            x \in \bigcap_{n=1}^\infty D_n = \bigcap_{n=1}^\infty \bigcap_{k=1}^n C_k = \bigcap_{n=1}^\infty C_n
        \]

        В частности, доказали непустоту пересечения $\bigcap_{n=1}^\infty C_n$ всех цилиндров $C_n$, а ровно это мы и хотели доказать.

        \item Нужно доказать, что функция $P$ счётно-аддитивна. В условиях задачи это эквивалентно непрерывности в нуле, то есть, что для любой убывающей к нулю последовательности множеств из $\cA$
        \[
            A_1 \supset A_2 \supset \dots,\ \bigcap_{n=1}^\infty A_n = \emptyset,\ \forall n \ A_n \in \cA
        \]
        выполнено $\lim\limits_{n \to \infty} P(A_n) = 0$. Зафиксируем произвольное $\eps > 0$. Знаем, что значения $P$ на множествах из $\cA$ приближаются с помощью компактного класса $\cK$, выберем для множеств $A_n$ приближение с увеличивающейся точностью, то есть выберем $K_n$ так, что $K_n \subset A_n$, $K_n \in \cK$ и $P(K_n) > P(A_n) - \frac{\eps}{2^n}$, что эквивалентно $P(A_n \setminus K_n) < \frac{\eps}{2^n}$.

        Так как $K_n \subset A_n$ и $\bigcap_{n=1}^\infty A_n = \emptyset$, то $\bigcap_{n=1}^\infty K_n = \emptyset$. В силу компактности класса $\cK$ есть $m$ для которого $\bigcap_{n=1}^m K_n = \emptyset$. Осталось применить не очень хитрый теоретико-множественный трюк, помня, что $A_1 \supset A_2 \supset \dots$:
        \[
            A_m = A_m \setminus \emptyset = A_m \setminus \ps{\bigcap_{n=1}^m K_n} = \bigcup_{n=1}^m (A_m \setminus K_n) \subset \bigcup_{n=1}^m (A_n \setminus K_n)
        \]

        И воспользоваться тем, что мы брали приближение с увеличивающейся точностью:
        \[
            P(A_m) \le \sum_{n=1}^m P(A_n \setminus K_n) \le \sum_{n=1}^m \frac{\eps}{2^n} \le \sum_{n=1}^\infty \frac{\eps}{2^n} = \eps
        \]
        Это и означает, что интересующий нас предел равен нулю.
    \end{enumerate}
\end{solution}